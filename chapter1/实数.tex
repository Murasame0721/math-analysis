\section{实数}

实数理论这一节的目标是从基本的集合论出发(只假设同学们熟知有理数和⾃然数),通过严格和完备的推理,证明若干论题作为基本的工具。这些通过严格论证得来的工具将会⽀撑着我们直观的图像,使得我们可以形象地思考和解决问题的同时不失严谨。

\subsection{实数的公理化描述}

$\RR$是一个集合,设$x,y,z\in \RR$

\begin{itemize}
    \item $\RR$上有两个操作
    
    \begin{itemize}
        \item 加法$+:\RR \times \RR \rightarrow \RR,(x,y)\mapsto x+y$
        \item 乘法:$\cdot \RR\times\RR\rightarrow\RR,(x,y)\mapsto x\cdot y$
    \end{itemize}

    \item $\RR$上有一个序关系$\leq$:$x\leq y$
\end{itemize}

规定四元组$(\RR,+,\cdot,\leq)$满足以下四套公理:

\begin{itemize}
    \item[(F)] 域公理:$\RR$是一个域
    \begin{enumerate}[(F1)]
        \item 加法结合律:$x+(y+z) = (x+y)+z$
        \item 加法交换律:$x+y=y+z$
        \item 存在加法单位元$\exists 0\in \RR.\forall x\in \RR,0+x=x$
        \item 加法逆元的存在性$\forall x\in \RR,\exists -x\in \RR,x+(-x)=0$\\
        \begin{remark}
            若(F1)-(F4)成立,则$(\RR,+)$被称作是一个交换群。同时,$-x$目前只是一个记号,因为我们还未定义减法运算。
        \end{remark}
        \item 乘法结合律$x\cdot(y\cdot z) = (x\cdot y)\cdot z$
        \item 乘法交换律$x\cdot y = y\cdot z$
        \item 存在乘法单位元$\exists 1\in \RR,1\neq 0,\forall x\in\RR, 1\cdot x = x$
        \item 乘法逆元的存在性$\forall x\in \RR-\{0\},\exists x^{-1}\in \RR,x\cdot x^{-1}=1$\\
        \begin{remark}
            (F5)-(F9)这四条公理表明:$(\RR^{\times}:=\RR\backslash\{0\},\cdot)$是一个交换群,同时强调$x^{-1}$也只是一个记号,因为还未定义乘方运算。
        \end{remark}
        \item 乘法分配律$x\cdot(y+z) = x\cdot y+x\cdot z$
    \end{enumerate}
\end{itemize}

假定满足(F1)-(F7)以及(F9),我们就称$(\RR,+,\cdot)$是一个交换环,满足这9条公理的$(\RR,+,\cdot)$被称作是一个域

对于正整数$n$,我们约定$x^n=x\cdot x\cdots \cdot x$(共$n$个$x$),$nx=x+x+\cdots +x$(共$n$个$x$)。类似的,对于$n\in\ZZ,n<0$,我们约定$x^n=x^{-1}\cdot x^{-1}\cdots \cdot x^{-1}$(共$n$个$x^{-1}$),$nx=(-x)+(-x)+\cdots+(-x)$(共$n$个$-x$)。我们规定$0x=0$,$x^0=1(x\neq 0)$.

由此,我们定义了以整数$n\in \ZZ$为幂的幂函数:
\[\RR\rightarrow\RR,x\mapsto x^n\]

\begin{itemize}
    \item[(O)] $\RR$是有序域
    \begin{enumerate}[(O1)]
        \item 序的传递性:$x\leq y,y\leq z\Rightarrow x\leq z$
        \item 序可以决定元素:$x\leq y,y\leq x\Rightarrow x=y$
        \item 全序关系$\forall x,y$,要么$x\leq y$,要么$y\leq x$(可以都成立)
        \item 与加法相容:$x\leq y\Rightarrow x+z\leq y+z$
        \item 与乘法相容:$x\geq 0,y\geq 0\Rightarrow xy\geq 0$
    \end{enumerate}
\end{itemize}

在$\geq,\leq$的基础上,我们给出大于号和小于号的定义:若$x\leq y$且$x\neq y$,则$x<y$,类似地,可以定义$x>y$.

由全序关系,我们可以知道,如下三种关系必居其一且互斥:

$x<y,x=y,x>y$

另外,若$x>0$,我们就称$x$是正实数并且称它的符号是正的,记作$\sign(x)=+$;如果$x<0$,我们就称$x$是负实数并且称它的符号是负的,记作$\sign(x)=-$.换句话说,我们定义了一个映射:
\[\sign:\RR^{\times} \rightarrow\{+,-\}\]

\begin{remark}
    注意这里的$\sign$并非$\sign$函数
\end{remark}

\begin{definition}[区间]
    设$a,b\in\RR,a<b$,定义开区间、闭区间和半开半闭区间如下:
    \begin{equation*}
        [a,b] :=\{x\in\RR : a\leq x\leq b\}
    \end{equation*}
    \begin{equation*}
        (a,b) :=\{x\in\RR : a< x< b\}
    \end{equation*}
    \begin{equation*}
        [a,b) := \{x\in\RR : a\leq x< b\}
    \end{equation*}
    \begin{equation*}
        (a,b] :=\{x\in\RR : a< x\leq b\}
    \end{equation*}
    另外,作为约定:
    \begin{equation*}
        [a,+\infty):={x\in\RR:x\geq a}
    \end{equation*}
    \begin{equation*}
        (a,+\infty):={x\in\RR:x< a}
    \end{equation*}
    \begin{equation*}
        (-\infty,b]:={x\in\RR:x\leq b}
    \end{equation*}
    \begin{equation*}
        (-\infty,b):={x\in\RR:x< b}
    \end{equation*}

    其中$a$和$b$分别被称作这些区间的左端点和右端点
\end{definition}

\begin{itemize}
    \item[(A)] Archimedes 公理:$\RR$是Archimedes有序域,即\\
    $\forall x>0$和$y$,$\exists n\in\NN, nx\geq y$
\end{itemize}

基于以上公理,我们可以定义整数。进一步地,我们可以定义有理数。

\begin{definition}[有理数]
    若$p,q\in\ZZ,q\neq0$,则定义$\displaystyle \frac{p}{q}$为有理数

    有理数集记作$\QQ$
\end{definition}

由有理数的定义,我们将$\RR\backslash\QQ$称为无理数

在实数这一章节中,一个关键点就是构造$\sqrt{2},e,\pi$这样的数。中学数学似乎从未给出这些数的具体定义(只是告诉了某些性质)。然而,上述公理并不足够定义这些无理数。

\begin{itemize}
    \item[(I)] 区间套公理
\end{itemize}

给定有限(即下面的$a_n$和$b_n$均为实数)闭区间的序列$\{I_n = [a_n,b_n]\}_{n=1,2,\cdots}$,如果这个序列是下降的,即$I_1\supset I_2\supset \cdots$,那么他们的交集非空,即

\begin{equation*}
    \lim_{n\rightarrow \infty}I_n:=\bigcap_{n\geq 1}I_n\neq \varnothing
\end{equation*}

\begin{definition}[实数]
    满足上述四套公理系统(F),(O),(A),(I)的四元组$(\RR,+,\cdot,\leq)$称作是实数
\end{definition}

事实上,有理数也满足(F),(O),(A)这三套公理系统,所以,要想得到我们中学所熟悉的实数,区间套公理是不可或缺的。

\subsection{Dedekind分割与实数的构造}

\begin{definition}[Dedekind分割]
    设集合$X\subset \QQ$,令$X' = \QQ \backslash X$,如果下面三条性质都成立:

    \begin{enumerate}
        \item $X\neq \varnothing,x'\neq \varnothing$
        \item $\forall x\in X,x'\in X$,都有$x<x'$
        \item $X$中没有最大元
    \end{enumerate}

    则称$X$或$X\cup X'$是$Q$的一个Dedekind分割。我们用$\mathcal{R}$表示所有Dedekind分割组成的集合
\end{definition}

\subsubsection*{序关系}

现在,我们要在$\mathcal{R}$上定义序关系,加法和乘法,使得$(\mathcal{R},+,\cdot,\leq)$满足四套公理。

首先定义序关系。对任意的$X,Y\in\mathcal{R}$,作为$\QQ$的子集,若

\begin{itemize}
    \item $X=Y$,则称$X=Y$
    \item $X\subset Y$,且$X\neq Y$,则称$X<Y$
\end{itemize}

不难验证这个序关系满足公理(O)

\subsubsection*{加法}

现在定义加法运算。对任意的$X,Y\in\mathcal{R}$,定义$X+Y:=\{x+y:x\in X,y\in Y\}$

同时定义零元素$\bar{0}$为$\bar{0}=X_0=\{x\in \QQ:x<0\}$

首先需要证明$X+Y$是个Dedekind分割,即

\begin{enumerate}
    \item $X+Y$及其补集是非空的
    \item $\forall x+y\in X+Y$,其中$x\in X,y\in Y$,如果有理数$z<x+y$,那么$z\in X+Y$
    \item $X+Y$没有最大元
\end{enumerate}

\begin{proof}
    先证明第一条。由于$X$和$Y$非空,显然,$X+Y$非空。

    随意选定$x_0'\in X',y_0'\in Y'$,那么$\forall x\in X,y\in Y$,都有$x<x_0',y<y_0'$
    
    所以,$\forall x\in X,y\in Y$,我们都有$x+y<x_0'+y_0'$,也就是说$x_0'+y_0'$比$X+Y$中所有数都大。
    
    特别地,$x_0'+y_0'\notin X+Y$,所以$X+Y$的补集不是空集

    再证明第二条。

    注意到$z<x+y\Leftrightarrow z-x<y$,所以$z-x\in Y$,又$x\in X$,所以$x+z-x\in X+Y$,即$z\in X+Y$

    最后证第三条。用反证法。

    假设$X+Y$中有最大元$x_0+y_0$,其中$x_0\in X,y_0\in Y$。根据定义,由于$X$和$Y$没有最大元,所以存在$x\in X,y\in Y$,使得$x_0<x,y_0<y$,从而$x_0+y_0<x+y$,而这与$x_0+y_0$是最大元的假设矛盾。
\end{proof}

下面,我们需要验证和加法相关的几条公理,如果不加其他的描述,默认$X,Y,Z\in\mathcal{R}$

根据定义,加法交换律(F2)和加法结合律(F1)显然成立。现证明(O4)

\begin{itemize}
    \item[(O4)] 如果$x<y$,那么$X+Z<Y+Z$ 
\end{itemize}

(O4)的证明比较复杂,需要用到一个引理

\begin{lemma}
    $X$是一个Dedekind分割。那么,对于任意正整数$n$,总存在$x\in X,x'\in X'$,使得

    \begin{equation*}
        0<x'-x,\frac{1}{n}
    \end{equation*}
\end{lemma}

\begin{proof}
    任意选定$x_0\in X,x_0'\in X'$,尝试归纳地构造一系列$(x_k,x_k')$,其中$x\geq 0,x_k \in X,x_k'\in X'$。

    设$(x_k,x_k')$已经构造完成,考虑$y=\displaystyle \frac{1}{2}(x_k+x_k')$,有两种情况:如果$y\in X$,那么令$(x_{k+1},x_{k+1}')=(y,x_k')$;如果$y\in X'$,那么定义$(x_{k+1},x_{k+1}')=(x_k,y)$

    显然,有

    \begin{equation*}
        |x_{k+1}-x_{k+1}'|=\frac{1}{2}|x_k-x_k'|\ \Rightarrow \ |x_k-x_k'| = \frac{1}{2^k}|x_0-x_0'|
    \end{equation*}

    对任意$k$成立。

    特别地,选取$k_0$,使得$2^{k_0}>n|x_0-x_0'|$,从而$x=x_{k_0}\in X$和$x'=x_{k_0}'\in X'$满足要求

    \qde
\end{proof}

现在回到(O4)的证明

\begin{proof}
    不难发现,若$X<Y$,显然$X+Z\leq Y+Z$.所以只需要证明$X+Z\neq Y+Z$即可

    在$Y\backslash X$中选取$y,\tilde{y}$.再选取合适的自然数$n$,使得$\displaystyle \frac{1}{n}<y-\tilde{y}$.不难证明,$n$总是存在。

    根据引理,再取$z\in Z,z'\in Z'$,使得$\displaystyle z'-x<\frac{1}{n}$

    现在,只需证明$y+z\notin X+Z$,使用反证法

    假设$y+z \in X+Z$,那么存在$x\in X,z_1\in Z$,使得$y+z=x+z_1$.

    按照定义$x<\tilde{y}$,所以$z_1<z'$.这样,就得到如下两个不等式:

    \begin{equation*}
        x<(\tilde{y}-y)+y
    \end{equation*}

    \begin{equation*}
        z_1<(z'-z)+z<\frac{1}{n}+z<(y-\tilde{y})+z
    \end{equation*}

    将两式相加,得到$x+z_1<y+Z$,矛盾
\end{proof}

接下来验证(F3)

\begin{itemize}
    \item[(F3)] $\forall X\in \mathcal{R},\bar{0}+X=X$
\end{itemize}

\begin{proof}
    根据$\bar{0}$的定义,$\bar{0}+X=\{x+(-a):a\in \QQ_{>0}\}$,
    
    假设$x-a\in \bar{0}+X$是任意给定的一个元素,其中$x\in X,a\in \QQ_{>0}$,那么,根据$x-a<x$,我们知道$x-a\in X$,即$\bar{0}+X\subset X$;

    假设$x\in X$是任给定的一个元素,那么一定存在$x'\in X$,使得$x>x'$

    令$a=x'-x\in\QQ_{>0}$,从而

    \begin{equation*}
        x=x'+(-a)
    \end{equation*}

    即$x\in \bar{0}+x$,所以$X\subset \bar{0}+X$
\end{proof}

\begin{itemize}
    \item[(F4)] 对任意的Dedekind分割$X$和$Y$,存在唯一的$Z\in \mathcal{R}$,使得$X+Z=Y$ 
\end{itemize}

$Y=\bar{0}$的情况就是公理(F5)

\begin{proof}
    $X,Y\in\mathcal{R}$给定,我们按照下面的方式定义$Z$
    \begin{equation*}
        Z=\{y-x':y\in Y,x' \in X' \}
    \end{equation*}

    首先需要证明$Z\in \mathcal{R}$,这个证明方法与之前验证$X+Y\in\mathcal{R}$的方法类似:

    \begin{itemize}
        \item $Z\neq \varnothing,Z' \neq \varnothing$ \\
        Z显然不是空集。为了说明$Z'$不是空集,我们选定$y_o'\in Y,x_0\in X$,那么只需要说明$y_0'-x_0\notin Z$即可;如若不然,存在$y\in Y$和$x'\in X'$,使得$y-x' = x_0'-x_0$,即$y+x_0=y_0'+x'$,这与$y<y_0'$和$x_0<x$矛盾
        \item 对任意的$y-x\in Z$,其中$y\in Y,x'\in X'$,如果$z<y-x'$,就一定有$z\in Z$\\
        只需要把$z$分解为$z=y-(y-z)$,根据$z<y-x'$,我们知道$y-z>x'$,我已$y-z\in X'$,所以$z$可以写成$Y$中元素与$X'$中元素的差
        \item $Z$中的元素没有最大元\\
        显然成立
    \end{itemize}

    现在来说明$X+Z=Y$

    由于$X+Z$中的元素形如$x+(y-x')$,其中$x\in X,y\in Y$,根据$x<x'$,所以这个元素必然大于$y$,即它在$Y$中,所以$X+Z\subset Y$

    对任意$Y$,由于$Y$没有最大元,可选取$\tilde{y}\in Y$,使得$y<\tilde{y}$.根据上面的引理,还可以选取$x\in X$和$x' \in X$,使得$x'-x<\tilde{y}-y$

    那么$\tilde{y}=y+(x'-x)<y$,所以$\tilde{y}\in Y$,从而
    \begin{equation*}
        y=x+(\tilde{y}-x')\in X+Z 
    \end{equation*}

    这说明$Y\subset X+Z$,综上,$X+Z=Y$
\end{proof}

由上面刚刚证明的(F4),我们可以定义相反数的运算:

\begin{equation*}
    -:\mathcal{R}\rightarrow\mathcal{R},X\mapsto -X
\end{equation*}

其中$-X$是使得$X+Z=\bar{0}$的那个$Z$

根据定义,我们有$-X=\{y-x':y\in \bar{0},x\in X'\}$,根据公理(F1)-(F4),这个$Z$是唯一的,并且$-(-X)=X$

同时,对于有理数$\displaystyle \frac{p}{q}$,有$X_{-\frac{p}{q}}=-X_{\frac{p}{q}}$,这个证明留作练习。

对于$X\in\mathcal{R}$,如果$X>\bar{0}$,我们就称$X$是正的;如果$X<\bar{0}$,我们就称$X$是负的。

\begin{example}%需要移动位置
    \begin{enumerate}
        \item 证明:$X$是正的当且仅当$X$中有正的有理数
        \item 证明:$X$是正的当且仅当$-X$是负的
    \end{enumerate}
\end{example}

根据例题的立一个结论,我们现在可以证明Archimedes公理(A),即:

如果Dedekind分割$X>\bar{0}$,那么对于任意的$Y\in\mathcal{R}$,存在正整数$n$,使得$nX>Y$(这里的$nX$意思是$n$个$X$相加)

类似于上面的证明,这个并不难验证

\subsubsection*{乘法}

现在定义乘法运算。对于$X,Y\in\mathcal{R}$,我们分情况讨论

\begin{equation*}
    X\cdot Y=\begin{cases}
        \bar{0},\quad \text{如果}X=\bar{0}\text{或}Y=\bar{0}\\
        \{x\cdot y:x\geq 0,x\in X,y\geq 0,y\in Y\}\cup\{0\},\quad\text{如果}X>0,Y>0\\
        -(X\cdot (-Y)),\quad\text{如果}X>0,Y<0\\
        -((-X)\cdot Y),\quad \text{如果}X<0,Y>0\\
        ((-X)\cdot(-Y)),\quad \text{如果}X<0,Y<0
    \end{cases}
\end{equation*}

我们希望(并将证明)乘法单位元恰好就是$\bar{1} = X_1 =\{x\in \QQ:X<1\}$

接下来的公理验证部分留作练习。

\section{实数的基数与不可数集}

首先,$\RR$是不可数集,这个我们之前其实已经证明过了,如果没有印象,请根据如下证明过程自行复习。

\begin{theorem}
    $\RR$是不可数集
\end{theorem}

\begin{proof}
    由命题\ref{th:repn}和无最大基数定理,得$\dbar{\RR}>\dbar{N}$.

    所以$\RR$是不可数集
\end{proof}

除上述证明外,还有Cantor对角线法则的经典证明。这里给出一种基于区间套原理的证明。

首先给出一个引理

\begin{lemma}
    如果$\{x_1,x_2,\cdots \}$是$\RR$的一个可数子集,那么存在闭区间套$I_1\supset I_2\supset \cdots$,使得对任意的$n$,$x_n \notin I_n$
\end{lemma}

\begin{proof}
    首先,显然能够找到一个闭区间,使得$x_1\notin I_1$,设$I_1=[a_1,b_1]$

    按照以下方法进行下一个闭区间的递推

    如果$\displaystyle x_{n+1}\in\left[a_n,\frac{a_n+b_n}{2}\right)$,则令$I_{n+1} = \displaystyle\left[\frac{a_n+b_n}{2},b_n\right]$;

    如果$\displaystyle x_{n+1}\in\left(\frac{a_n+b_n}{2},b_n\right]$,则令$\displaystyle I_{n+1} = \left[a_n,\frac{a_n+b_n}{2}\right]$;

    如果$\displaystyle x_{n+1}=\frac{a_n+b_n}{2}$,则令$\displaystyle I_{n+1} = \left[a_n,\frac{a_n+b_n}{4}\right]$;

    如果$x_{n+1}\notin I_n$,则令$I_{n+1}=I_n$

    这样,就得到了一个闭区间套,使得$\forall n,x_n\notin I_n$

    \qde
\end{proof}

现在正式开始证明$\RR$是不可数集

\begin{proof}
    假设$\RR$是$\RR$的一个可数子集,那么存在一个闭区间套$I_1\supset I_2\supset \cdots$,使得对任意的$n$,$x_n \notin I_n$.

    但是,对于任意的闭区间,总有一个实数元素,这与假设矛盾,所以$\RR$不可数。

    \qde
\end{proof}

\begin{definition}
    称$\RR$的基数为连续基数,记为$c$(或$\aleph_1$)
\end{definition}

类似于可数集的可数并仍然可数,基数为$c$的集合经过可数并后,依然基数为$c$

\begin{theorem}
    设集合列$\{A_k\}$,若$\forall k,\dbar{A_k}=c$,则
    \begin{equation*}
        A = \bigcup_{k=1}^{\infty}A_k
    \end{equation*}

    的基数依然为$c$
\end{theorem}

\begin{proof}
    不妨假定$A_k$之间交集为空。设$A_k\sim[k,k+1)$,有

    \begin{equation*}
        \bigcup_{k=1}^{\infty}A_k\sim[1,\infty)\sim \RR
    \end{equation*}
    \qde
\end{proof}