\section{实数}

实数理论这一节的目标是从基本的集合论出发(只假设同学们熟知有理数和⾃然数),通过严格和完备的推理,证明若干论题作为基本的工具。这些通过严格论证得来的工具将会⽀撑着我们直观的图像,使得我们可以形象地思考和解决问题的同时不失严谨。

\subsection{实数的公理化描述}

$\RR$是一个集合,设$x,y,z\in \RR$

\begin{itemize}
    \item $\RR$上有两个操作
    
    \begin{itemize}
        \item 加法$+:\RR \times \RR \rightarrow \RR,(x,y)\mapsto x+y$
        \item 乘法:$\cdot \RR\times\RR\rightarrow\RR,(x,y)\mapsto x\cdot y$
    \end{itemize}

    \item $\RR$上有一个序关系$\leq$:$x\leq y$
\end{itemize}

规定四元组$(\RR,+,\cdot,\leq)$满足以下四套公理:

\begin{itemize}
    \item[(F)] 域公理:$\RR$是一个域
    \begin{enumerate}[(F1)]
        \item 加法结合律:$x+(y+z) = (x+y)+z$
        \item 加法交换律:$x+y=y+z$
        \item 存在加法单位元$\exists 0\in \RR.\forall x\in \RR,0+x=x$
        \item 加法逆元的存在性$\forall x\in \RR,\exists -x\in \RR,x+(-x)=0$\\
        \begin{remark}
            若(F1)-(F4)成立,则$(\RR,+)$被称作是一个交换群。同时,$-x$目前只是一个记号,因为我们还未定义减法运算。
        \end{remark}
        \item 乘法结合律$x\cdot(y\cdot z) = (x\cdot y)\cdot z$
        \item 乘法交换律$x\cdot y = y\cdot z$
        \item 存在乘法单位元$\exists 1\in \RR,1\neq 0,\forall x\in\RR, 1\cdot x = x$
        \item 乘法逆元的存在性$\forall x\in \RR-\{0\},\exists x^{-1}\in \RR,x\cdot x^{-1}=1$\\
        \begin{remark}
            (F5)-(F9)这四条公理表明:$(\RR^{\times}:=\RR\backslash\{0\},\cdot)$是一个交换群,同时强调$x^{-1}$也只是一个记号,因为还未定义乘方运算。
        \end{remark}
        \item 乘法分配律$x\cdot(y+z) = x\cdot y+x\cdot z$
    \end{enumerate}
\end{itemize}

假定满足(F1)-(F7)以及(F9),我们就称$(\RR,+,\cdot)$是一个交换环,满足这9条公理的$(\RR,+,\cdot)$被称作是一个域

对于正整数$n$,我们约定$x^n=x\cdot x\cdots \cdot x$(共$n$个$x$),$nx=x+x+\cdots +x$(共$n$个$x$)。类似的,对于$n\in\ZZ,n<0$,我们约定$x^n=x^{-1}\cdot x^{-1}\cdots \cdot x^{-1}$(共$n$个$x^{-1}$),$nx=(-x)+(-x)+\cdots+(-x)$(共$n$个$-x$)。我们规定$0x=0$,$x^0=1(x\neq 0)$.

由此,我们定义了以整数$n\in \ZZ$为幂的幂函数:
\[\RR\rightarrow\RR,x\mapsto x^n\]

\begin{itemize}
    \item[(O)] $\RR$是有序域
    \begin{enumerate}[(O1)]
        \item 序的传递性:$x\leq y,y\leq z\Rightarrow x\leq z$
        \item 序可以决定元素:$x\leq y,y\leq x\Rightarrow x=y$
        \item 全序关系$\forall x,y$,要么$x\leq y$,要么$y\leq x$(可以都成立)
        \item 与加法相容:$x\leq y\Rightarrow x+z\leq y+z$
        \item 与乘法相容:$x\geq 0,y\geq 0\Rightarrow xy\geq 0$
    \end{enumerate}
\end{itemize}

在$\geq,\leq$的基础上,我们给出大于号和小于号的定义:若$x\leq y$且$x\neq y$,则$x<y$,类似地,可以定义$x>y$.

由全序关系,我们可以知道,如下三种关系必居其一且互斥:

$x<y,x=y,x>y$

另外,若$x>0$,我们就称$x$是正实数并且称它的符号是正的,记作$\sign(x)=+$;如果$x<0$,我们就称$x$是负实数并且称它的符号是负的,记作$\sign(x)=-$.换句话说,我们定义了一个映射:
\[\sign:\RR^{\times} \rightarrow\{+,-\}\]

\begin{remark}
    注意这里的$\sign$并非$\sign$函数
\end{remark}

\begin{definition}[区间]
    设$a,b\in\RR,a<b$,定义开区间、闭区间和半开半闭区间如下:
    \begin{equation*}
        [a,b] :=\{x\in\RR : a\leq x\leq b\}
    \end{equation*}
    \begin{equation*}
        (a,b) :=\{x\in\RR : a< x< b\}
    \end{equation*}
    \begin{equation*}
        [a,b) := \{x\in\RR : a\leq x< b\}
    \end{equation*}
    \begin{equation*}
        (a,b] :=\{x\in\RR : a< x\leq b\}
    \end{equation*}
    另外,作为约定:
    \begin{equation*}
        [a,+\infty):={x\in\RR:x\geq a}
    \end{equation*}
    \begin{equation*}
        (a,+\infty):={x\in\RR:x< a}
    \end{equation*}
    \begin{equation*}
        (-\infty,b]:={x\in\RR:x\leq b}
    \end{equation*}
    \begin{equation*}
        (-\infty,b):={x\in\RR:x< b}
    \end{equation*}

    其中$a$和$b$分别被称作这些区间的左端点和右端点
\end{definition}

\begin{itemize}
    \item[(A)] Archimedes 公理:$\RR$是Archimedes有序域,即\\
    $\forall x>0$和$y$,$\exists n\in\NN, nx\geq y$
\end{itemize}

基于以上公理,我们可以定义整数。进一步地,我们可以定义有理数。

\begin{definition}[有理数]
    若$p,q\in\ZZ,q\neq0$,则定义$\displaystyle \frac{p}{q}$为有理数

    有理数集记作$\QQ$
\end{definition}

由有理数的定义,我们将$\RR\backslash\QQ$称为无理数

在实数这一章节中,一个关键点就是构造$\sqrt{2},e,\pi$这样的数。中学数学似乎从未给出这些数的具体定义(只是告诉了某些性质)。然而,上述公理并不足够定义这些无理数。

\begin{itemize}
    \item[(I)] 区间套公理
\end{itemize}

给定有限(即下面的$a_n$和$b_n$均为实数)闭区间的序列$\{I_n = [a_n,b_n]\}_{n=1,2,\cdots}$,如果这个序列是下降的,即$I_1\supset I_2\supset \cdots$,那么他们的交集非空,即

\begin{equation*}
    \lim_{n\rightarrow \infty}I_n:=\bigcap_{n\geq 1}I_n\neq \varnothing
\end{equation*}

\begin{definition}[实数]
    满足上述四套公理系统(F),(O),(A),(I)的四元组$(\RR,+,\cdot,\leq)$称作是实数
\end{definition}

事实上,有理数也满足(F),(O),(A)这三套公理系统,所以,要想得到我们中学所熟悉的实数,区间套公理是不可或缺的。