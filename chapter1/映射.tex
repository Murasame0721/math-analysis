\section{映射与基数}

通过映射,我们把不同的集合联系起来。而运用映射联系不同的集合时,有时会出现不能建立联系的情况,这种情况发生的原因是因为集合中元素的多少不同。

对于有限集,我们很容易描述它元素的个数来表示其中元素的多少;而对于无限集,元素的多少是难以描述的。同为无限集时,有些集合的元素远多于另一个集合以至于不能一一对应。但有时,看起来元素不一样多的集合却能够形成一一对应的关系(比如说整数和偶数)。为了衡量集合中元素的多少,引入基数的概念。

\subsection{映射}

在中学中,我们已经学过函数的概念。函数是从定义域到$\RR$的一种对应关系,我们把这种概念推广到一般的集合。

\begin{definition}[映射]
    设非空集合$X,Y$,若$\forall x\in X$,存在唯一的$y\in Y$与之对应,则称这个对应为映射。若用$f$表示这种对应,则记作

    \begin{equation*}
        f:X\rightarrow Y
    \end{equation*}

    并称$f$是从$X$到$Y$的一个映射。
\end{definition}

类似于复合函数,有复合映射

\begin{definition}
    设$f:X\rightarrow Y,g:Y\rightarrow W$,则
    \[h(x) = g(f(x)),x\in X \]

    定义的$h$称为$g$与$f$的复合映射,可记作$f\circ g$
\end{definition}

类似于函数的自变量和因变量,映射有“像”和“原像”。

\begin{definition}[像、原像]
    设$x\in X$,$y$是$Y$中与$x$对应的元素,称$y=f(x)$是$x$的像,$x$是$y$的原像。
    
    对于集合,称$f(A)=\{y\in Y:x\in A,y=f(x)\}$为$A$的像集(可简称为像);

    称$f^{-1}(B)=\{x\in X:y\in B,y=f(x)\}$为$B$关于$f$的原像集(可简称为原像)。
\end{definition}

显然,像具有以下性质

\begin{enumerate}
    \item $\displaystyle f\left(\bigcup_{\alpha \in I}A_\alpha \right)= \bigcup_{\indexAlpha}f(A_\alpha)$
    \item $\displaystyle f\left(\bigcap_{\alpha \in I}A_\alpha \right)\subset \bigcap_{\indexAlpha}f(A_\alpha)$
    \item $\displaystyle f^{-1}\left(\bigcup_{\alpha \in I}B_\alpha \right)= \bigcup_{\indexAlpha}f^{-1}(B_\alpha)(B_\alpha \subset Y,\indexAlpha)$
    \item $\displaystyle f^{-1}\left(\bigcap_{\alpha \in I}B_\alpha \right)= \bigcap_{\indexAlpha}f^{-1}(B_\alpha)(B_\alpha\in Y,\indexAlpha)$ 
    \item 若$B_1\subset B_2 \subset Y$,则$f^{-1}(B_1)\subset f^{-1}(B_2)$
\end{enumerate}

接下来,将要讲到3个重要的映射——单射、满射、双射。

\begin{definition}[单射]
    不同元有不同像的映射是单射,即:

    设$f:X\rightarrow Y$,若$\forall x_1,x_2 \in X\text{且}x_1 \neq x_2$时,有$f(x_1)\neq f(x_2)$,则$f$是单射。
\end{definition}

    一种单射的等价描述是:任何一个像都只有一个原像的映射是单射

\begin{definition}[满射]
    $Y$是像的映射是满射,即:

    设$f:X\rightarrow Y$,若$\forall y\in Y,\exists x\in X,y=f(x)$,则$f$是满射。
\end{definition}

\begin{definition}[双射]
    既是单射又是满射的映射是双射
\end{definition}

不难发现,整数集和偶数集能够形成双射。事实上,与真子集形成双射是无限集区别于有限集的一个重要特征。

\begin{proposition}
    一个集合能够与它的真子集形成双射当且仅当它是无限集。
\end{proposition}

\begin{proof}
    先证明充分性。这个命题的逆否命题是:如果一个集合是有限集,那么它不能够和它的真子集形成双射。

    有限集$E$的元素个数为$n$,假设$E$的真子集$A$与$E$形成双射$f:E\rightarrow A$。

    因为$f$是双射,$\forall a\in A$,总能找到不重复的$b\in E$使得$f(b) = a$

    因为$E$有$n$个元素,所以能在$A$中找到$n$个不重复的元素,这与$A\subsetneq E$矛盾,所以假设不成立,充分性得证。

    证明必要性只需要找到一个任何无限集都存在的到自身真子集的双射。

    设无限集$B$,由于$B$是无限集,必定存在一个自然数集$\NN$到$B$的单射,于是,得到了一列元素${b_n}(n\in \NN+)$.

    这样,可以定义映射$g:B\rightarrow B\backslash\{b_1\}$,$g$的定义如下:

    若$\forall n\in \NN+,x\neq b_n$,则令$g(x) = x$;若$\exists  n_0\in\NN+,x = b_{n_0}$,则令$g(x) = b_{n_0+1}$

    显然,$g$是双射,必要性得证
\end{proof}

在下一部分中,我们将比较集合的多少。比较集合的多少会经常需要证明两个集合之间存在双射。

\subsection{基数}

对于一个集合,集合中元素的个数是最基本的问题之一,设集合$A,B$,如何比较哪一个集合元素的个数多?

对于有限集,比较集合元素的多少是简单的。但对于无限集,情况就复杂了。首先我要告诉你:无限集的元素并不是一样多的,不要因为不知道怎么比较就认为无限集的元素都一样多。

首先,一个经典且有趣的问题,自然数和偶数哪个多?

你也许会想当然地认为:自然数包含偶数和奇数,所以自然数多。但是,我们可以轻易地把自然数和偶数一一对应(只需要乘2就可以),这时,它们看起来是一样多的。

数学上要求,比较集合元素的多少用第二种方法——利用映射(尤其是双射)进行比较。

\begin{definition}
    设集合$A,B$,若存在一个双射$f:A\rightarrow B$,则称集合$A$与$B$对等,记作$A\sim B$
\end{definition}

\begin{example}
    求证$\NN \times \NN \sim \NN$
\end{example}

\begin{proof}
    我们知道,任何一个大于1的整数都能够被素因数分解。并且,2是最小的素数。

    所以,任意自然数$n$均可唯一地表示为$n=2^p \cdot q$,其中$p$为非负整数,$q$为正奇数。

    于是,就存在双射
    \[f(i,j) = 2^{i-1}(2j-1),\quad (i,j\in \NN) \]
    
    满足$\NN \times \NN \rightarrow \NN$

\end{proof}

显然,对等关系具有以下基本性质:

\begin{enumerate}
    \item $A\sim A$
    \item 若$A\sim B$,则$B\sim A$
    \item 若$A\sim B,B\sim C$,则$A\sim C$
\end{enumerate}

由这些性质,我们就可以通过过渡集合获得两个集合之间的一一对应关系。此外,还可以采用分解与合成的思想,尤其是以下的Cantor-Bernstein定理,更是一种证明集合对等的重要手段。

\begin{remark}
    这个定理的证明是一个难点
\end{remark}

\begin{theorem}[Cantor-Bernstein定理]
    若集合$X$与$Y$的某个真子集对等,$Y$与$X$的某个真子集对等,则$X\sim Y$

    这个定理的一种等价叙述是:若集合$X,Y$间存在映射$f:X\rightarrow Y,g:Y\rightarrow X$,且都是单射,则$X\sim Y$
\end{theorem}

\begin{proof}
    首先对集合进行划分。设单射$f:X\rightarrow Y,g:Y\rightarrow X$

    定义一个可能用于划分集合的过程$r$:

    对集合$X$中的点$x\in A$,$r(x) = g^{-1}(x) \in Y$;

    对集合$B$中的点$y\in B$,$r(y) = f^{-1}(y) \in X$.

    对$X$中的点进行$r$过程,得到$Y$中的点,再对这个点进行$r$过程,会再次得到$X$中的点(假设这些操作均可行)

    注意到,集合$X\backslash g(Y)$和集合$Y\backslash f(X)$中的点不能进行$r$过程。

    根据是否能无限地迭代进行$r$过程,将$X,Y$分别分割成3个部分,即
    \[ X= X_\infty \sqcup X_X \sqcup X_Y\]
    \[Y = Y_\infty \sqcup Y_X \sqcup Y_Y\]

    其中下标为$\infty$的集合中的点能够无限地进行$r$过程;下标为$X$的集合在进行$n(n\geq 0)$次过程后,落入集合$X\backslash g(Y)$;下标为$Y$的集合在进行$n(n\geq 0)$次过程后,落入集合$Y\backslash f(X)$

    首先,限制在$X_\infty \rightarrow Y_\infty$的映射$f$是一个双射,理由如下:

    对任意一个$Y_\infty$中的点$y_\infty$,进行一次$r$过程,必定会得到$X_\infty$中的点$x_\infty$,即
    \[\forall y_\infty \in Y_\infty,\ \exists x_\infty ,\ f^{-1}\left(y_\infty \right) = x_\infty\]
    
    所以,$f:X_\infty \rightarrow Y_\infty$是满射。

    又因为$f$是单射,所以$f:X_\infty \rightarrow Y_\infty$是双射。

    同理,$f: X_X \rightarrow Y_X$是双射。

    对于$X$的最后一部分($X_Y$),我们不能仿照上面的做法。因为对于$Y \backslash f(X) \subset Y_Y$,而$Y \backslash f(X)$是$f$是达不到的。

    所以对于这一部分,我们要用$g$构造双射。

    对任意$X_Y$中的点$x_Y$,进行一次过程$r$,必然会得到$Y_Y$中的点$y_Y$,即

    \[\forall x_Y \in X_Y,\ \exists y_Y\in Y_Y,\ g^{-1}(x_Y) = y_Y \]

    所以$g: Y_Y\rightarrow X_Y$是双射,所以$g^{-1}:X_Y \rightarrow Y_Y$是双射。

    于是,就得到了双射$h:X\rightarrow Y$

    \begin{equation*}
        h(x) =
    \begin{cases}
         f(x), \text{当}x\in X_\infty \cup X_X \\
         g^{-1}(x),\text{当}x\in X_Y
    \end{cases}
    \end{equation*}

\end{proof}

\begin{remark}
    符号$\sqcup$称作无交并,这个符号的意义是:对于满足$M\cap N = \varnothing$的集合$M,N,$有$M\sqcup N = M\cup N$
\end{remark}