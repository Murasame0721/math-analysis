\chapter{集合}

在中学阶段,大家已经初步接触过关于集合的知识。例如,自然数全体形成一个集合,常记为$\NN$;有理数形成一个集合,常记为$\QQ$;实数形成一个集合,记为$\RR$.关于集合的精确定义是很难给出的,根据Cantor给出的概念(概括性),可以这样定义集合。

\begin{definition}
    集合是把具有某种特征或满足一定性质的所有对象视为整体时,这个整体是集合,而这些对象就称为集合中的元素
\end{definition}

在这个描述性的定义上,需要建立公理,满足数学的严谨性要求。

在集合的基础上,我们可以为集合定义距离、范数、向量内积、二元运算。通过定义运算,可以形成代数结构;通过定义距离、范数、向量内积,可以形成各种空间。

在对集合有基础的了解后,我们将在最常用的集合——实数集$\RR$上,定义序关系、加法、乘法,并开始我们的数学分析学习之旅。

\section{集合与子集}

我们约定,集合的符号用大写字母$A,B,C,\cdots,X,Y,Z$等表示,集合中的元素用$a,b,c,\cdots,x,y,z$等表示。若$a$是$A$的元素,则记为$a \in A$,称$a$属于$A$

对于集合,在中学阶段我们已经学过了这些定义

\begin{definition}[子集]
    对于两个集合$A,B$.若$x \in A$必定有$x \in B$,则称$A$是$B$的子集 ,记作:
    \[ A\subset B \]
    如果$\exists b \in B, b \notin A$,则称$A$是$B$的真子集,记作$A\subsetneq B$
\end{definition}

\begin{definition}[空集]
    空集是不包含任何元素的集合,记作$\varnothing $

    规定:空集是任何集合的子集
\end{definition}

\begin{definition}
    设集合$A,B$,若$A\subset B$且$B \subset A$,则称$A$与$B$相等或等同,记作$A=B$
\end{definition}

对于一系列具有共同特征的集合,我们可以对每一个集合进行标号,可以将标号组成的集合记作$I$,称为指标集。在此基础上,可以给出集合族的定义

\begin{definition}[集合族]
    设$I$是给定的一个集合,对于每一个$\indexAlpha$,指定一个集合$A_\alpha$ ,这样可以得到一系列集合,它们的总体称为集合族,记为$\left\{A_\alpha : \indexAlpha \right\}$或者$\left\{A_\alpha \right\}_{\indexAlpha}$

    当$I=\NN$时,集合族也称为集合列,简记为$\left\{A_i \right\}$这样的形式

    集合族常常用花体字母表示,如$\mathcal{A},\mathcal{B} ,\mathcal{P} $ 
\end{definition}

事实上,上述集合的描述是不完美的。我们可以构造出一种情况,使得一个元素既不能属于一个集合,又不能不属于这个集合。

\begin{example}[罗素悖论]
    定义$S=\left\{A:A\notin A \right\}$,判断$S\in S$是否成立
\end{example}

首先,我们需要明白的是,上面的$\notin$并不是$\nsubset$,这里不是作者笔误(这是初学者常有的误解)。

为了便于理解$A \in A$是什么情况,我们可以先尝试找到一个符合这种性质的集合。事实上,我们不难发现:由全体无限集组成的集合满足$A\in A$。换句话说,全体无限集组成的集合属于它自身。

回到正题。假设$S\in S$,那么根据$S$的定义,有$S \notin S$;假设$S \notin S$,则根据定义,有$S \in S$.从而无法判断$S$是否是$S$的元素

这个悖论类似于理发师悖论。感兴趣的同学可以上网搜索。

罗素悖论和它引申出的其他悖论要求对集合设置自洽的公理体系,著名的公理系统有ZF公理系统和NBG公理系统。这些内容超过了本书涉及范围,故不赘述。

\section{集合的运算}

集合的分解和合成是形成新集合的有效方法,这种分解和合成可以通过集合间的运算来表达。

在中学阶段,我们已经学过简单的集合的并、交、补。

\subsection{交与并}

\begin{definition}
    设集合$A,B$,称集合$\left\{x: x\in A \text{或} x \in B \right\}$ 为$A$与$B$的并集,记作$A \cup B$.
\end{definition}

\begin{definition}
    设集合$A,B$,称集合$\left\{x: x\in A \text{且} x \in B \right\}$ 为$A$与$B$的交集,记作$A \cap B$.

    若$A \cap B = \varnothing$,则称$A$与$B$互不相交。
\end{definition}

显然,集合的运算满足结合律、交换律、分配律

\begin{theorem}
 
    
    交换律:
    \begin{equation}
        A\cup B = B \cup A,\ A\cap B = B\cap A;
    \end{equation}

    结合律:
    \begin{equation}
    \begin{split}
        A\cup (B\cup C) = (A \cup B) \cup C ,  \\  
        A\cap (B\cap C) = (A \cap B) \cap C;
    \end{split}
    \end{equation}

    分配律:
    \begin{equation}
    \begin{split}
        A\cap (B\cup C) = (A\cap B)\cup (A\cap C), \\
        A\cup (B\cap C) = (A\cup B)\cap (A\cup C);
    \end{split}
    \end{equation}
\end{theorem}

类似于两个集合的交、并,可以定义多个集合的交、并。(此处的多个可以是无穷多个,甚至可以是比自然数的个数更大的无穷)

\begin{definition}
    设集合族$\left\{A_\alpha \right\}_{\indexAlpha}$,定义并集和交集如下:

    \begin{equation*}
        \bigcup_{\indexAlpha}A_\alpha = \left\{ x: \exists \indexAlpha, x \in A_\alpha \right\} 
    \end{equation*}

    \begin{equation*}
        \bigcap_{\indexAlpha} A_\alpha = \left\{ x:\forall \indexAlpha, x\in A_\alpha \right\}
    \end{equation*}
\end{definition}

根据上述定义,对于任意多(可以是无穷多)个的并或交,改变计算顺序不会影响结果。

\begin{remark}
上述结论并非通过数学归纳法和两个集合的交、并的性质得出。数学归纳法只能保证“任意有限”成立,对“无限”不能保证成立。
\end{remark}

分配律对于多个集合的运算依然成立。

\begin{theorem}
    多个集合运算的分配律:
    \begin{equation}
    \begin{split}
        A\cap \left(\bigcup_{\indexAlpha} B_\alpha\right) = \bigcup_{\indexAlpha}(A\cap B_\alpha); \\
        A\cup \left(\bigcap_{\indexAlpha} B_\alpha\right) = \bigcap_{\indexAlpha}(A\cup B_\alpha).
    \end{split}
    \end{equation}
\end{theorem}

\begin{example}[康托尔集]
    取闭区间$[0,1]$,挖去中间的$1/3$,得到两个部分(闭区间),再分别对两个部分挖去中间,得到4个部分。不断地进行挖去中间的操作,重复无穷多次后,得到的集合就是康托尔集。请尝试用交集和并集表示康托尔集。
\end{example}

第一次挖去中间时,得到的两个闭区间是$\displaystyle \left[0,1/3\right],\left[2/3,1\right]$,第二次挖去中间时,挖去的是$\left[1/9,2/9\right],\left[7/9,8/9\right]$.注意到分母总是$3^n$,并且保留部分区间下限的分子始终是偶数,区间上限的分子始终是下限+1.于是,康托尔集可以写成这样的形式

\begin{equation*}
    \bigcap_{k=0}^{\infty} \bigcup_{i=0}^{3^k} \left[\frac{2i}{3^k},\frac{2i+1}{3^k}\right]
\end{equation*}

\subsection{差与补}

\begin{definition}[差集]
    设集合$A,B$,称$\left\{x:x\in A, x\notin B \right\}$为$A$与$B$的差集,记作$A\backslash B$.(读作$A$减$B$)

    当$B \subset A$时,称$A\backslash B$为集合$B$相对于$A$的补集(或余集)。
\end{definition}

在讨论某一问题时,常常规定一个默认的最大集合$X$,我们称$X$为全集,此时,集合$B$相对于全集的补集就简称$B$的补集,并记作$B^c$

显然,有如下事实:

\begin{enumerate}
    \item $A\cup A^c = X, A\cap A^c = \varnothing, (A^c)^c = A$
    \item $X^c = \varnothing,\varnothing^c = X$
    \item $A\backslash B = A\cap B^c$
    \item 若$A\supset B$,则$A^c\subset B^c$
    \item 若$A\cap B = \varnothing$,则$A\subset B^c$
\end{enumerate}

集合的补与交、并有如下运算法则

\begin{theorem}[De.Morgan法则]
    \begin{equation}
        \label{eq:deMorgan1}
        \left(\bigcup_{\indexAlpha} A_\alpha\right)^c = \bigcap_{\indexAlpha} A_\alpha^c
    \end{equation}
    \begin{equation}
        \label{eq:deMorgan2}
        \left(\bigcap_{\indexAlpha} A_\alpha\right)^c = \bigcup_{\indexAlpha} A_\alpha^c
    \end{equation}
\end{theorem}

\begin{proof}
    以\eqref{eq:deMorgan1}为例,对于任意$\displaystyle x\in \left(\bigcup_{\indexAlpha} A_\alpha\right)^c$,有$\displaystyle x\notin \bigcup_{\indexAlpha}A_\alpha$,即$\forall \indexAlpha,x\notin A_\alpha$

    也就是说$\forall \indexAlpha,x\in A^c_\alpha$,故$\displaystyle x\in \bigcap_{\indexAlpha}A_\alpha^c$

    所以,$\displaystyle \left(\bigcup_{\indexAlpha} A_\alpha\right)^c \subset \bigcap_{\indexAlpha} A_\alpha^c$

    反过来,对于任意$\displaystyle x\in \bigcap_{\indexAlpha}A_\alpha^c$,有$\forall \indexAlpha,x\in A^c_\alpha$,即$\forall \indexAlpha,x\notin A_\alpha$

    所以$\displaystyle x\in \left(\bigcup_{\indexAlpha} A_\alpha\right)^c$

    所以$\displaystyle \bigcap_{\indexAlpha} A_\alpha^c \subset \left(\bigcup_{\indexAlpha} A_\alpha\right)^c$

    综上,\eqref{eq:deMorgan1}得证
\end{proof}

请读者模仿上述证明过程,自行证明\eqref{eq:deMorgan2}

\begin{remark}
    证明两个集合互为子集是证明两个集合相等的一种常用方法。这种方法也可以推广到其他与“子集”相关的证明,如数论中证明两个整数相等可以用互相整除证明。
\end{remark}

\subsection*{集合列的极限}

本部分将在极限章节中讲述

\subsection{笛卡尔积}

笛卡尔积(Cartesian product)又称直积,是一种将多个集合中的元素直接组合起来的运算。

\begin{definition}
    设集合$X,Y$,$x\in X,y\in Y$,称一切形如$(x,y)$的有序“元素对”形成的集合为$X,Y$的笛卡尔积,记作$X\times Y$,即

    \begin{equation*}
        X\times Y=\left\{(x,y):x\in X,y\in Y \right\}
    \end{equation*}

    集合对自身的笛卡尔积如$X\times X$可以记为$X^2$
\end{definition}

\subsection{幂集}

\begin{definition}
    设$X$是一个非空集合,由$X$的一切子集(包括$\varnothing$和自身)为元素形成的集合称为$X$的幂集,记作$\PP(X)$
\end{definition}

\begin{example}
    设$E$是有$n$个元素的有限集,求$\PP(E)$的元素个数
\end{example}

对于任何一个元素,要么它在$E$的子集中,要么不在。并且,一个元素是否在子集不影响另外一个元素是否在子集中。所以,$\PP(E)$的元素个数为$2^n$

从直观上看,一个集合的幂集的元素个数一定比这个集合本身的元素个数多。事实上,对于无限集来说也是如此,在之后的“基数”章节会给出严格的证明。