\chapter{集合}

在中学阶段,大家已经初步接触过关于集合的知识。例如,自然数全体形成一个集合,常记为$\mathbb{N}$;有理数形成一个集合,常记为$\mathbb{Q}$;实数形成一个集合,记为$\mathbb{R}$.关于集合的精确定义是很难给出的,根据Cantor给出的概念(概括性),可以这样定义集合

\begin{definition}
    集合是把具有某种特征或满足一定性质的所有对象视为整体时,这个整体是集合,而这些对象就称为集合中的元素
\end{definition}

在这个描述性的定义上,需要建立公理,满足数学的严谨性要求。

\section{集合与子集}

我们约定,集合的符号用大写字母$A,B,C,\cdots,X,Y,Z$等表示,集合中的元素用$a,b,c,\cdots,x,y,z$等表示。若$a$是$A$的元素,则记为$a \in A$,称$a$属于$A$

对于集合,在中学阶段我们已经学过了这些定义

\begin{definition}
    对于两个集合$A,B$.若$x \in A$必定有$x \in B$,则称$A$是$B$的子集,记作:
    \[ A\subset B \]
    如果$\exists b \in B, b \notin A$,则称$A$是$B$的真子集,记作$A\subsetneq B$
\end{definition}

\begin{definition}
    空集是不包含任何元素的集合,记作$\varnothing $

    规定:空集是任何集合的子集
\end{definition}

\begin{definition}
    设集合$A,B$,若$A\subset B$且$B \subset A$,则称$A$与$B$相等或等同,记作$A=B$
\end{definition}

对于一系列具有共同特征的集合,我们可以对每一个集合进行标号,可以将标号组成的集合记作$I$,称为指标集。在此基础上,可以给出集合族的定义

\begin{definition}
    设$I$是给定的一个集合,对于每一个$\alpha \in I$,指定一个集合$A_\alpha$ ,这样可以得到一系列集合,它们的总体称为集合族,记为$\left\{A_\alpha : \alpha \in I \right\}$或者$\left\{A_\alpha \right\}_{\alpha \in I}$

    当$I=\mathbb{N}$时,集合族也称为集合列,简记为$\left\{A_i \right\}$这样的形式

    集合族常常用花体字母表示,如$\mathcal{A},\mathcal{B} ,\mathcal{P} $ 
\end{definition}

事实上,上述集合的描述是不完美的。我们可以构造出一种情况,使得一个元素既不能属于一个集合,又不能不属于这个集合。

\begin{example}
    (罗素悖论)定义$S=\left\{A:A\notin A \right\}$,判断$S\in S$是否成立
\end{example}

首先,我们需要明白的是,上面的$\notin$并不是$\nsubset$,这里不是作者笔误(这是初学者常有的误解)。

为了便于理解$A \in A$是什么情况,我们可以先尝试找到一个符合这种性质的集合。事实上,我们不难发现:由全体无限集组成的集合满足$A\in A$。换句话说,全体无限集组成的集合属于它自身。

回到正题。假设$S\in S$,那么根据$S$的定义,有$S \notin S$;假设$S \notin S$,则根据定义,有$S \in S$.从而无法判断$S$是否是$S$的元素

这个悖论类似于理发师悖论。感兴趣的同学可以上网搜索。

罗素悖论和它引申出的其他悖论要求对集合设置自洽的公理体系,著名的公理系统有ZF公理系统和NBG公理系统。这些内容超过了本书涉及范围,故不赘述。